\documentclass[a4paper]{article}
\usepackage[letterpaper,top=2cm,bottom=2cm,left=2cm,right=2cm,marginparwidth=1.75cm]{geometry}

\usepackage[UTF8]{ctex}
\usepackage{float}
% Useful packages
\usepackage{braket}
\usepackage{color}
\usepackage{amsmath}
\usepackage{graphicx}
\usepackage[colorlinks=true, allcolors=blue]{hyperref}
\usepackage{listings}
\lstset{language=C++}%这条命令可以让LaTeX排版时将C++键字突出显示
\lstset{breaklines}%这条命令可以让LaTeX自动将长的代码行换行排版
\lstset{extendedchars=false}%这一条命令可以解决代码跨页时,章节标题,页眉等汉字不显示的问题

\title{ 机器学习笔记整理}
\begin{document}
\maketitle
\begin{abstract}
	这份文档是我们在学习机器学习相关知识的同时,用来记录讨论所得的文档。这份记录将会和《动手学深度学习》这本书的章节相匹配。
\end{abstract}
\section{Chapter 2.1-2.3 May 24}
\begin{enumerate}
	\item 在PyTorch库中使用的数据类型张量,除了可以简单地进行按元素的加减乘除等运算之外,在不同形状的张量之间还可以通过来按元素操作。广播机制的可以自动扩展张量为相等大小。
	如果遵守以下规则,则两个tensor是“可广播的”:
\begin{enumerate}
    \item 每个tensor至少有一个维度;
    \item 遍历tensor所有维度时,从末尾随开始遍历,两个tensor存在下列情况:
        \\tensor维度相等。
        \\tensor维度不等且其中一个维度为1。
        \\tensor维度不等且其中一个维度不存在。
\end{enumerate}
 如果两个tensor是“可广播的”,则计算过程中,如果两个tensor的维度不同,则在维度较小的tensor的前面增加维度,使它们维度相等。对于每个维度,计算结果的维度值取两个tensor中较大的那个值。tensor扩展维度是指将数值进行复制到扩展后的维度上。
\\具体例子可以参考:https://zhuanlan.zhihu.com/p/402163854

\item 在使用PyTorch库中的张量数据类型时,需要注意节省内存。在机器学习中,我们可能有数百兆的参数,并且在一秒内多次更新所有参数。通常情况下,我们希望原地执行这些更新。原地更新的办法有:
\begin{enumerate}
	\item 用切片表示法将操作的结果分配给先前分配的数组,如
	\begin{lstlisting}
Z = torch.zeros_like(Y)
Z[:] = X + Y
	\end{lstlisting}
	这样不会为变量Z分配新的内存,而是会改变原有地址处的内容。
	\item	使用X[:] = X + Y或X += Y,而非X = X + Y,最后一种办法会改变变量X所指向的地址,相当于重新找了一个内容空间来存放求和的结果。≠≠
\end{enumerate}

\end{enumerate}


\end{document}